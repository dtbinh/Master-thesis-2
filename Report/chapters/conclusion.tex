\chapter{Conclusion and Outlook}

\section{Conclusion}
The goal of this project was to extend the existing LMPC strategy to repetitive tasks and test this theory on the race driving problem.\\
We were able to show that - by only little extensions - the LMPC strategy is also applicable to continuous repetitive systems. The resulting controller is feasible and stable. In our application of race driving we could also show that the iteration cost is non-increasing. Simulations and experiments on a real car showed that it is real time feasible and converges to a locally optimal solution.\\
Our control strategy does not need any offline computations beforehand and works in real time - even for a complex system with 6 states and long prediction horizons of 30 time steps.\\
We developed an experimental setup using a 1:10 scale remote-controlled race car (BARC) that is easy to transport and set up at different locations.\\
We presented an online system identification method that allowed us to optimize the trajectory without prior knowledge of the system's parameters. And even though we were using low-cost sensors we were able to develop an estimator that provides reliable state estimates of sufficient quality for race driving.\\
In experiments we reached velocities of up to $\SI{3}{\meter\per\second}$ on a racetrack with maximum curvatures of $\SI{1.2}{\per\meter}$ which corresponds - according to the scaling factor - to maximum velocities of $\SI{100}{\kilo\meter\per\hour}$ and a minimum curve radius of $\SI{8}{\meter}$.\\
A strong advantage of using the LMPC strategy in combination with online system identification is that the model can slowly be learned more and more accurate from iteration to iteration while decreasing the iteration cost at the same time. This makes it possible to optimize a repetitive task without prior knowledge of the system parameters. However, the system's dynamics have to be known in order to construct a linear regression model.\\
Certain assumptions and approximations still remain a challenge. Using polynomials to approximate the curvature and the safe set did not allow to use longer time horizons, as the polynomials would have to be more complex to fit these functions. Eventually, a tradeoff between the MPC horizon length, the complexity of the polynomials (mainly determined by their degrees), and the solving time of the solver has to be found.
%\begin{itemize}
%\item Extended previous theory for repetitive LMPC
%\item Found a method to gradually find an optimal trajectory even in a complex system
%\item Applied on self-driving car
%\item Real-time feasible, no previous offline calculations necessary
%\item Used online system identification with Linear Regression
%\item Complex state estimation, no camera
%\item High curvature in the test track
%\item What might pose problems in implementation: Solving the mixed integer nonlinear programming problem in real time. Our proposed approach of relaxing the terminal constraints and cost function works kind of well but can lead to problems if the safe set trajectories can't be well approximated by low order polynomials. Higher order polynomials, however, lead to solver difficulties (very high first derivatives, many local optima).
%\end{itemize}

\section{Outlook and future work}
While we were assuming no model mismatch in simulations of the LMPC strategy, we clearly had some model mismatch in real world experiments. How to handle this model mismatch and what assumptions and assertions can still be made in this scenario is subject of current research.\\
Also, so far the LMPC solution converges to a local optimum but we don't know a method to find a global optimum. Developing a systematic approach that finds a global optimum (e.g. by artificially disturbing the solution trajectory) is a topic that could improve this method.\\
Another idea that can lead to faster convergence to the optimal solution is to use symmetries of state dynamics within iterations. As in the car racing example, this would lead to similar treatment of similar track sections (e.g. deceleration before, acceleration in curves).\\
This work already made some simplifications to the LMPC problem (e.g. relaxing the terminal set, approximating the curvature by polynomials). However, the resulting LMPC is still a complex nonlinear problem which remains a challenge to solve efficiently and in real time. Further simplifications or transformations of this problem might allow even shorter solution times and therefore faster reaction times.\\
% Estimation:
In order to improve lap times of the BARC experimental setup even more and to allow higher velocities, the estimation system would need to be improved. One major difficulty proved to be the proper estimation of the car's heading angle. A very promising method would be to use a stationary camera system (e.g. from \cite{Liniger2015}). Using markers on different points on the car would allow measuring not only the current position of a car but also its heading angle. Also using an onboard camera that detects lane boundaries or even the entire surroundings (e.g. Simultaneous Localization and Mapping, SLAM) might be used in the estimation, although this may be computationally expensive.\\

%\begin{itemize}
%\item BARC specific:
%\begin{itemize}
%\item Improvement of the state estimation approach
%\item Add a camera that detects the surroundings and/or track bounds. E.g. SLAM might work, needs high computational power though (maybe refer to another paper).
%\item Use better encoders
%\end{itemize}
%\item Race driving theory specific
%\begin{itemize}
%\item Adding obstacles
%\item Multiple car race driving (moving obstacles)
%\item Use recurring patterns in state dynamics to learn faster (e.g. left curves, obstacles, ...)
%\end{itemize}
%\end{itemize}